% \title{Praca Inżynierska}
% \author{Jacek Eichler \\ David Solomon \\ Stanisław Kucznerowicz}

\documentclass[12pt,oneside,a4paper]{book} % jednostronnego


\usepackage{polski}
\usepackage[utf8]{inputenc} %opcja dla edytorów kodujących polskie znaki w utf8
%\usepackage[cp1250]{inputenc} %opcja dla edytorów kodujących polskie znaki w windows-1250
\usepackage{lmodern}
\usepackage{indentfirst}
\usepackage{microtype}
\DisableLigatures{encoding = *, family = * }
\usepackage{fancyhdr}
\usepackage{pstricks,graphicx}
\usepackage{amssymb}
\def\bibname{Literatura}% zmienia nazwę Bibliografia na literatura


%---------------Zbiory liczbowe
\newcommand{\R}{\mathbb{R}}
\newcommand{\N}{\mathbb{N}}
\newcommand{\K}{\mathbb{K}}
\newcommand{\C}{\mathcal{C}}
\newcommand{\p}{\mathcal{P}}
%------------kwantyfikatory--------------
\newcommand{\fal}{\mbox{{\Large $\forall\,$}}}
\newcommand{\ext}{\mbox{{\Large $\exists\,$}}}
%------------------definicje środowisk-----------------
\usepackage{theorem}
\theoremstyle{break}
\theorembodyfont{\it}
\newtheorem{twr}{Twierdzenie}[chapter]
\newtheorem{lem}{Lemat}[chapter]
\theorembodyfont{\rm}
\newtheorem{defi}{Definicja}[chapter]
\newtheorem{wni}{Wniosek}[chapter]
\newtheorem{prz}{Przykład}[chapter]
\newenvironment{dowod}{\par\vspace{0.1cm}\par{ \sc Dowód.}}{\hfill $\blacksquare$\par\vspace{0.4cm}\par}
% ----------ustawienia wymiarow strony
\usepackage{geometry}

\newgeometry{tmargin=2.5cm, bmargin=2.5cm, headheight=14.5pt, inner=3cm, outer=2.5cm} 

\linespread{1.1} %-zmiana interlinii

%---------------- Normalne środowiska --------------------
\usepackage{amsmath}

%----------nagłowki i żywa pagina ------------
\pagestyle{fancy} 

%--------------- Wydruk jednostronny
\fancyhead[C]{} 
\fancyfoot[C]{\thepage}
\fancyhead[L]{\scriptsize\leftmark}
\fancyhead[R]{\scriptsize\rightmark}

\renewcommand{\chaptermark}[1]{%
\markboth{\MakeUppercase{%
\chaptername}\ \thechapter.%
\ #1}{}}

\renewcommand{\sectionmark}[1]{\markright{\thesection.\ #1}}


\usepackage{lipsum} %------ Można usunąć przy pisaniu pracy
%-----------------właściwa część pracy-----------------
\begin{document}
\thispagestyle{empty}
\begin{center}{\sc \Large
Politechnika Poznańska \\
{\large
WYDZIAŁ INFORMATYKI I TELEKOMUNIACJI \\
Instytut informatyki \\ 
Kierunek Informatyka
}
\end{center}
\vspace{5cm}
\begin{center}
{\LARGE
APLIKACJA MOBILNA DO ROZPOZNAWANIA ROŚLIN \\
}\par\vspace{1cm}\par
{\large
Jacek Eichler \\ David Solomon \\ Stanisław Kucznerowicz
}
\end{center}
\vspace{4cm}
\begin{flushright}
%praca dyplomowa\\
praca inżynierska\\
napisana pod kierunkiem\\
dr inż. Marek Kraft
\end{flushright}
\vfill
\begin{center}
Poznań 2021
\end{center}

\newpage

% \vspace*{16cm}
\chapter*{Streszczenie}

Aktualnie na całym świecie można zanotować tendencje wzrostową rynku roślin doniczkowych. Bardzo często gdy kupujemy bądź dostajemy w ramach prezentu jakąś roślinę, to nie wiemy tak naprawdę jakiego jest ona rodzaju. W związku z tym właściciele takich nierozpoznanych roślin, nie wiedzą jaką przyjąć stategię przy ich pielęgnacji. Kolejnym problemem jest sam proces indetyfikacji, gdyż w przypadku braku informacji o typie rośliny, należałoby skorzystać z wiedzy eksperckiej lub książki florystycznej, aby ją rozpoznać. \par \vspace{0.2cm}
Próbą rozwiązania tego problemu ma być aplikacja mobilna służąca do identyfikacji roślin. Jej użytkownicy poprzez wykonanie zdjęcia przy jej użyciu, otrzymają odpowiedź z nazwą rodzaju rośliny. Kolejną funkcjonalnością aplikacji jest informacja o zalecanej pielegnacji zidentyfikowanej rośliny. Wszystkie wykonane klasyfikacje roślin będą dostępne użytkownikowi. Dzięki temu, wybrane rośliny będą mogłby być dodane do specjalnej grupy, która reprezentuje rośliny w jego domu. Aplikacja będzie inteligetnie informowała  użytkownika o konieczności pielęgnacji rośliny.

\thispagestyle{empty} \setcounter{page}{0} \tableofcontents



\thispagestyle{empty}

\chapter{Wprowadzenie}
\thispagestyle{empty}

\pagestyle{fancy}
\chapter{Część teoretyczna}
\thispagestyle{empty}

\pagestyle{fancy}
\chapter{Założenia i sformułowanie wymagań}
\thispagestyle{empty}
\pagestyle{fancy}
\chapter{Architektura systemu}
\thispagestyle{empty}
\pagestyle{fancy}
\chapter{Dokumentacja projektowa}
\thispagestyle{empty}
\pagestyle{fancy}
\chapter{Dokumentacja programowa}
\thispagestyle{empty}
\pagestyle{fancy}
\chapter{Testowanie i integrowanie oprogramowania}
\thispagestyle{empty}
\pagestyle{fancy}
\chapter{Dokumentacja użytkowa}
\thispagestyle{empty}
\pagestyle{fancy}
\chapter{Podsumowanie}
\thispagestyle{empty}


\begin{thebibliography}{00}
\addcontentsline{toc}{chapter}{Literatura}

\end{thebibliography}

\end{document}